\documentclass{article}
\begin{document}
John F. Lake, Jr. 
CSE 20189
HW 8 #1
\begin{itemize}
\item	autoconf - Generates configuration scripts based on the configure.ac file.  This would be helpful for making sure source code compiles the correct way in a given environment.  
\item	automake - Used to create individual makefiles, based on what the configure script outputs.  This is useful for setting information about how to compile in a given environment. 
\item   libtool - Used to create portable libraries. This is helpful if your program includes a lot of code from libraries. 
\item	gettext - Used for creating multi-language messages.  This is helpful if you are distributing your code to other countries with different languages.  
\item 	pkg-config - Helps you insert correct compiler options with your code.  This is helpful for not hardcoding your compiler options.  


\end{itemize}


\end{document}
