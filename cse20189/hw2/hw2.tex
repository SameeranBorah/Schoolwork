\documentclass{article} 

\begin{document} 

John F. Lake, Jr.  
CSE 20189 
HW #2

\section{}
/usr/bin/script creates a transcript of everything on the terminal window, and this can be printed out later. This would be useful to record some of the things a student may do in a terminal.  

\section{}

Currently Loaded Modulefiles:

\begin{tabular}{l c r}
  1) modules     &          4) intel/10.0      &      7) opt_local/1.0 \\
  2) matlab/7.14   &        5) maple/12       &       8) schrodinger/2007 \\
  3) pgi/8.0       &        6) mathematica/7.0    &   9) initialize/standard \\
\end{tabular}

\section{}
/sbin/ifconfig

\section{}
Use the command :q!$<$enter$>$. 

\section{}
\begin{tabular}{l c r}
Host Name & IP Address & MAC Address \\
student00.cse.nd.edu & 129.74.152.73 & 00:25:B3:E0:4D:C6 \\
student01.cse.nd.edu & 129.74.152.74 & 00:25:B3:E3:B7:FC \\
student02.cse.nd.edu & 129.74.152.75 & 00:25:B3:E3:B8:04   \\
student03.cse.nd.edu & 129.74.152.76 & 00:25:B3:E0:4B:FC \\
\end{tabular}

\section{} 
The -C flag in ssh and scp enables compression of files.  This would be useful if you are uploading large files, or are handling a lot of data at one time.  

\section{}
I would use ping and traceroute. To use these tools, I would type either ping $<$server\_name$>$ or traceroute $<$server\_name$>$. I would use these tools when I have confirmed there is no error on my end because they are used when there is a problem with the remote service (the server) rather than something on my end.  

\section{}
The first 15 characters are ``AAAAB3NzaC1yc2E''.

\section{}
In \cite{Quigley2005},
regular expressions are defined as as "a pattern of characters used to match the same characters in a search."
\bibliographystyle{plain}
\bibliography{bibliography}

 

\end{document} 
